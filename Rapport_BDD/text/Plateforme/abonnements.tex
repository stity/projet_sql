\subsection{Gestion des abonnements}
Cette partie va présenter toutes les fonctionnalités relatives à la consultation, la modification et la souscription à un abonnement. Du point de vue des utilisateurs, tous (y compris les visiteurs) ont accès à la consultation des offres (formules et forfaits étrangers). Par contre, un client à accès en plus à un résumé de ses achats (téléphone et formules).

\subsubsection{La liste des formules}
Cette vue est la principale associée à la gestion des abonnements. Elle présente un résumé de la liste des formules proposés par Centrale-Télécom. Le tableau présente donc les principales caractéristiques des formules, à savoir :
\begin{itemize}
  \itemperso{Nom et tarif}Les informations basiques pour que les utilisateurs se repèrent dans les formules.
  \itemperso{Promotion}Cette valeur est mise à \og Oui\fg{} si la formule est une promotion, \og Non\fg{} sinon.
  \itemperso{Téléphone associé}L'icône permet de savoir si la formule est associée à un ou plusieurs téléphones. Ceci permet de représenter les liens (par exemples promotionnels) entre une formule et un téléphone. Si un ou deux téléphones sont associés, alors l'icône est \vColor{\faMobilePhone}, sinon il s'agit de \thColor{\faMobilePhone}.
  \itemperso{Forfaits étrangers}Tout comme le champs sur les téléphones, il permet de savoir si l'offre est associée à une formule vers l'étranger. Ici aussi, si un ou plusieurs forfaits étrangers sont inclus dans la formule, alors l'icône est \vColor{\faGlobe}, sinon il s'agit de \thColor{\faGlobe}.
  \itemperso{Boutons d'édition}
\end{itemize}
Ces différents éléments peuvent être observés à la Figure~\ref{fig:}


%%% Local Variables:
%%% mode: latex
%%% TeX-master: "../../Rapport_BDD"
%%% End:
