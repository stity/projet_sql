\nnsection{Conclusion}
En conclusion, ce travail nous a permis d'appliquer une partie des nouveauté abordées lors de ce module. En particulier, notre application a essayé de tirer le meilleur partie de l'utilisation des procédures. Ainsi, ces dernières nous ont permis de stocker la plupart de nos requêtes en base de données, et ainsi de faciliter leur appel dans le code.

Pour le code, le choix a été d'utiliser une technologie vue lors du S5 de Centrale Lyon, à savoir d'utiliser le langage PHP en se connectant à une base de données SQL en utilisant \texttt{mysqli}.

Outre les aspects SQL, ce projet fut également une occasion d'implémenter une utilisation de feuilles de XSL pour réutiliser des données en provenance d'une base de données MySQL et de les mettre en forme (ici sous forme de PDF).

Dans un dernier temps, ce projet nous a permis d'essayer de voir comment l'utilisation de SQL pouvait s'intégrer dans la mise en place d'une architecture MVC pour une application/plateforme web.

Ce projet fut donc une occasion de tester plusieurs technologies et principes et de voir leurs interractions. Si certains aspects de la plateforme auraient pu être plus développé, nous avons ici pris le partie de tester des solutions diverses, afin de voir les possibilités des langages et leurs relations.

%%% Local Variables:
%%% mode: latex
%%% TeX-master: "../../Rapport_dreches"
%%% End:
