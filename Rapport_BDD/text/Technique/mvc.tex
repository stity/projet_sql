\subsection{Architecture de la plateforme}

\subsubsection{Architecture MVC}
La plateforme a été réalisée en ayant pour volonter de mettre en place une architecture MVC minimale.

Pour cela, le code a été organisé selon trois dossiers principaux :
\begin{itemize}
  \itemperso{\texttt{controleur}}Ce fichier contient tous les contrôleurs associés au modèle.
  \itemperso{\texttt{vue}}Ce fichier contient toutes les vues du site, ie. des visuels qui réutilisent des informations calculées par les contrôleurs.
  \itemperso{\texttt{database}}Normalement, le fichier aurait dû être appelé \texttt{modèle} dans une architecture MVC classique, cependant, toutes les opérations liées au modèle sont presque exclusivement réalisées en SQL dans cette application, ce fichier a donc \og pris sa place\fg.
\end{itemize}
Dans la suite, nous allons préciser comment les ressources sont chargées pour chaque page.

\subsubsection{Appel des vues}
\subParagraphe{Charger les fichiers}Le centre de l'architecture mise en place est le fichier \texttt{index.php}. Ce dernier réalise toutes les inclusions nécessaires en PHP, via les commandes :

\begin{lstlisting}[language=php]
  require 'database/DB.php';
  require 'controleur/users_controller.php';
  require 'controleur/bills_controller.php';
  require 'vue/vue.php';
  .../...
\end{lstlisting}

\subParagraphe{Création d'une vue}Ensuite, nous créons dans ce fichier une vue (instance de l'objet \texttt{Vue}). Cet instance permettra d'afficher les éléments dans la page.

\subParagraphe{Assignation de la vue à un contrôleur}Enfin, le fichier va récupérer via les paramètres passés dans l'URL, le nom de la vue qui est demandée. En fonction du résultat, il va créer le contrôleur associé à la vue. Ceci est réalisée par le \texttt{swicth} suivant :

\begin{lstlisting}[language=php]
  if (isset($_GET['vue'])){
    switch($_GET['vue']){
      case 'users':
        $controleur = new UsersController($vue, $_SESSION['log_in'], $_SESSION['login_level']);
        break;
      case 'bills':
        $controleur = new BillsController($vue, $_SESSION['log_in'], $_SESSION['login_level']);
        break;
      case .../...
\end{lstlisting}

Ainsi, chaque contrôleur est crée avec la vue pour l'afficher en paramètre. Chaque contrôleur va alors appeler spécifiquement son affichage via des commandes de la forme :

\begin{lstlisting}[language=php]
  function __construct($vue)  {
    $vue->display('bills', 'Factures', 'Consulter ma facture', $this);
  }
\end{lstlisting}
% $

\subParagraphe{Récupération du fichier spécifique à la vue}Enfin, la dernière étape revient à charger le fichier de la vue associée au contrôleur (et non plus la classe \texttt{Vue}). Pour cela, l'instance \texttt{\$vue} dispose d'une méthode \texttt{determine\_vue} qui est appelée par \texttt{display} et qui contient les associations entre les vues demandées et leur fichier associé.

\subParagraphe{Relancer le processus}Ce processus sera relancé à partir de la recherche du contrôleur dès que le client clique sur un des liens du panneau latéral.

\subsubsection{Les fichiers affichés}
Comme nous l'avons évoqué, les fichiers affichés sont situés dans le répertoire \texttt{vue}. De manière plus précise, le fichier contenant la bare latéral de navigation et qui est affiché quelle que soit la situation est nommé \texttt{main.html.php}. Tous les autres fichiers ont des noms relatifs aux objets donc ils sont les vues. Ainsi, \texttt{phone.html.php} correspond par exemple à la vue pour les téléphones.

Comme cela a été expliqué, toutes les correspondances sont visibles dans le fichier \texttt{vue.php} qui contient la classe \texttt{Vue}.

\subsubsection{Scripts, styles et autres éléments}
\subParagraphe{Feuilles de style}Les feuilles de styles de l'application sont stockées dans le dossier \texttt{assets/styles}.
\subParagraphe{Scripts JS}Les scripts Javascript utilisés pour l'application sont stockées dans le dossier \texttt{assets/scripts}.
\subParagraphe{Polices}L'application utilise avec parcimonie la police Font Awesome pour afficher des éléments graphiques (\faChevronDown, \faUser, \faRemove,...). Cette police est contenue dans le dossier \texttt{assets/fonts}.
\subParagraphe{Fichiers XSL et XML}Ces fichiers utilisés pour la génération du PDF (voir Section~\ref{sec:pdf}) sont directement disposés dans le dossier \texttt{vue}, car directement liés à ces dernières.

%%% Local Variables:
%%% mode: latex
%%% TeX-master: "../../Rapport_BDD"
%%% End:
