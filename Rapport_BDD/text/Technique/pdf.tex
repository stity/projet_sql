\subsection{La génération du PDF de consommation détaillée}
Pour pouvoir générer le PDF nous avons besoins de plusieurs choses :
\begin{enumerate}
	\item Récupérer les informations dans la base de données
	\item Générer un fichier XML avec ces informations
	\item Associer ce fichier XML à une feuille de style XSLT qui permet de transformer le document XML en document XSL:FO.
	\item Générer le fichier avec FOP
	\item Rediriger l'utilisateur vers le PDF.
\end{enumerate}

\subsubsection{Récupérer la consommation dans la base de données}
Les données doivent être triées selon l'ordre chronologique.
Cette opération est réalisée au moyen de la requête SQL suivante :


\begin{lstlisting}[language=sql]
SELECT date,volume, nom as destination, zone FROM sms JOIN (
	SELECT pays.id,pays.nom as nom, zone_geographique.nom as zone FROM pays, zone_geographique, zone_geographique_pays
	WHERE pays.id=zone_geographique_pays.pays AND zone_geographique.id=zone_geographique_pays.zone_geographique
	)
pays ON (sms.destination=pays.id) WHERE consommation=consoId ORDER BY date ASC;
\end{lstlisting}

\subsubsection{Génération du fichier XML}

A partir des informations précédemment récupérée, on vient remplir un fichier XML avec le script PHP partiel suivant :

\begin{lstlisting}[language=php]
$result = $db->execute($sql);
        if ($result->num_rows > 0) {
            while($row = $result->fetch_assoc()) {
                $date = new DateTime($row['date']);
        echo "<sms>\n";
            echo "<date>".$date->format('d/m \à H\hi' )."</date>\n";
            echo "<volume>".$row['volume']."</volume>\n";
            echo "<destination>".$row['destination']."</destination>\n";
            echo "<zone>".$row['zone']."</zone>\n";
        echo "</sms>\n";
\end{lstlisting}

Ce qui donne le fichier XML suivant :

\begin{lstlisting}[language=xml]
	<?xml version="1.0" encoding="utf-8"?><?xml-stylesheet type="text/xsl" href="conso.xsl"?><conso>
<generationdate>22/01 à 21:06:09</generationdate><smslist>
<sms>
<date>02/11 à 12h42</date>
<volume>6</volume>
<destination>Rwanda</destination>
<zone>Reste du monde</zone>
</sms>
...
</smslist>
<mmslist>
<mms>
<date>01/11 à 18h38</date>
<volume>3</volume>
<destination>Argentine</destination>
<zone>Amérique du Sud</zone>
</mms>
...
</mmslist>
<appellist>
<appel>
<date>01/11 à 12h49</date>
<duree>00:28:57</duree>
<destination>Arabie Saoudite</destination>
<zone>Reste du monde</zone>
</appel>
...
</appellist>
</conso>
\end{lstlisting}


\subsubsection{La transformation en PDF}
La feuille de style XSLT ne sera pas expliquée ici car ce n'est pas le sujet principal. Néanmoins, le lecteur curieux pourra la trouver dans le fichier \texttt{conso.xsl}. Il s'agit de venir peupler trois tableaux XSL:FO avec les données XML.

On transforme ensuite cette combinaison XML et XSLT en PDF grâce à l'utilitaire Apache FOP qui est écrit en JAVA.
La génération est lancée par le script suivant (lequel est appelé par PHP).

\begin{lstlisting}[language=bash]
cd fop/build
java -jar fop.jar -xml ../../conso.xml -xsl ../../conso.xsl -pdf ../../conso.pdf
cd ../..
\end{lstlisting}

Il suffit ensuite de rediriger la page vers le fichier PDF.

\begin{lstlisting}[language=php]
header('Location: conso.pdf');
\end{lstlisting}

