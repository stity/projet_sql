\subsection{Sujet traité}
Le sujet que nous avons choisi de traiter est celui du site d'un compagnie de téléphone mobile. Ce dernier vous est rappelé ci-dessous :
\vspace*{.3cm}
\newline
\og{\itshape La compagnie de téléphonie mobile Centrale-Telecom en concurrence directe avec les services Orange, SFR et BouyguesTelecom, offre différents types de forfaits au public. Chaque formule est proposée à un certain prix et varie en nombre d’heures de communication par mois et en options. Les options concernent :
  \begin{itemize}
    \itemperso{}Un certain nombre de SMS par mois,
    \itemperso{}L'accès internet,
    \itemperso{}Un volume de données transférées par internet(en ko),
    \itemperso{}Des plages horaires d'appels illimités,
    \itemperso{}Plus toutes les options que vous voudrez ajouter pour écraser la concurrence.
  \end{itemize}
  Par ailleurs une offre de téléphones existe, chaque téléphone étant d'une part caractérisé par ses caractéristiques (écran, capacités internet, TV, appareil photo, video numérique, mémoire,...) et d'autre part peut être éventuellement associé à une offre à un prix promotionnel.

Pour ce qui est de la facturation, on gèrera les communications selon la tarfication par forfait (locale, nationale et internationale par pays) sachant qu'un forfait comprend un certain montant mensuel en communications. On n'oubliera pas les communications hors forfait. On gèrera également la facturation
des autres services (SMS, Internet, MMS). Enfin la facturation sera associée à l'éventuel règlement ou non de la facture ainsi que le blocage éventuel d'une ligne en cas d'impayés.

Le travail à réaliser consiste à concevoir une base de données gérant les différents forfaits et la clientèle, ainsi qu'offrir des informations sur les forfaits, les promotions du mois et éventuellement consultation de l'état de la consommation des clients.}\fg.

%%% Local Variables:
%%% mode: latex
%%% TeX-master: "../../Rapport_BDD"
%%% End:
