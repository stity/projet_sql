Afin d'avoir un fonctionnement plus proche d'un vrai site de téléphoniste, les améliorations suivantes seraient envisageables :
\begin{itemize}
  \itemperso{Mécanisme de commande}Le mécanisme de commande actuel est assez rudimentaire. Il serait intéressant que ce processus se fasse par des formulaires successifs qui enregistrent les choix des clients au fur et à mesure et propose un résumé à la fin à valider. Ce processus permettrait également de créer des achats avec formule et téléphones groupés, ce qui n'est aujourd'hui pas possible (à moins d'écrire la requête à la main!).
  \itemperso{Recherches avancées}Si nous avons mis pour l'exemple un mécanisme de recherche avancée pour la table des abonnements, il serait intéressant de généraliser (et améliorer) ce fonctionnement aux autres tables, avec des requêtes SQL créées à la volée pour remplir les conditions attendues.
  \itemperso{Tri dans les tables}Un autre élément d'ergonomie intéressant aurait été d'offrir la possibilité aux utilisateur de réaliser des tris dans les tables (pas prix, par promotion,...). Ceci n'a pas été mis en place dans les interfaces, bien que cela soit mis en place pour la génération du PDF de facture, qui montre donc un exemple de réalisation que nous n'avons pas eu le temps de généraliser à toute l'application.
  \itemperso{Dynamisme}De manière générale, l'interface (bien qu'utilisant du Javascript), reste assez statique, un travail pourraît donc être à réaliser dans ce sens. L'aspect graphique et ergonomique pourrait mériter plus de travail, nous avons cependant préférer réaliser les fondamentaux nécessaires à pouvoir utiliser l'application dans le périmètre fixé initiallement.
\end{itemize}



%%% Local Variables:
%%% mode: latex
%%% TeX-master: "../../Rapport_BDD"
%%% End:
