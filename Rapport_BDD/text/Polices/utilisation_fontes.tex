\subsection{Utilisation des polices sous XeLaTeX}
L'apparition de XeLaTeX et LuaLaTeX a simplifié l'utilisation des polices pour les utilisateurs de \LaTeX, dans la mesure où toute police installée sur le système peut être utilisée par ces compilateurs en utilisant le package \texttt{fontspec} pour charger la police.

Par défaut, XeLaTeX et LuaLaTeX suppose que les textes sont écrits en UTF-8, il ne faut donc pas charger le package \texttt{inputenc} en les utilisant, et les fichiers sources peuvent être sauvegardés en UTF-8.

Ainsi, un document de base aura l'aspect suivant :

\begin{lstlisting}[language={[LaTeX]TeX}]
  \documentclass{article}

  \usepackage{fontspec}
  \setmainfont{<any font on your system>}

  \begin{document}
  ...
  \end{document}
\end{lstlisting}

\subsubsection{Indiquer la police par défaut du document}
Pour un document entier, on peut spécifier :
\begin{itemize}
  \itemperso{}La police principale en style \og roman\fg.
  \itemperso{}La police principale pour le style \og sans empattement\fg.
  \itemperso{}Le police principale pour le style \og largeur fixe\fg.
\end{itemize}
Les commandes à utiliser sont les suivantes :

\begin{lstlisting}[language={[LaTeX]TeX}]
  \setmainfont[<font features>]{<name of font>} % police roman
  \setsansfont[<font features>]{<name of font>} % police sans empattements
  \setmonofont[<font features>]{<name of font>} % police avec largeur fixe
\end{lstlisting}

\subsubsection{Utiliser une police locallement}
Pour l'utilisation d'une police dans une partie précise du document, le mieux est de définir une nouvelle famille de polices.

\begin{lstlisting}[language={[LaTeX]TeX}]
  \newfontfamily\myfont[<font features>]{<name of font>}
\end{lstlisting}

Ceci créé un déclencheur pour la police, appelé \texttt{\textbackslash myfont} qui permet de passer à la police désignée. Il est également possible de sélectionner directement une fonte avec la commande \texttt{\textbackslash fontspec}, cependant ceci n'est pas conseillé, dans la mesure où la méthode \texttt{\textbackslash newfontfamily} est bien plus efficace.

\subsubsection{Les possibilités des polices}
Comme le package \texttt{fontspec} fournit des accès aux polices de type Open Type, il est possible d'accéder à des nombreuses fonctionalités particulières de ces polices. Ces fonctions peuvent être sélectionnées en utilisant les arguments optionnels pour la sélection de n'importe quelle police. Pour plus de détails, se référer à la documentation du package \texttt{fontspec}, les plus courantes sont :
\begin{itemize}
  \itemperso{\texttt{[Ligatures=TeX]}}Cela permet d'utiliser les ligatures \TeX{} usuelles (qui ne sont pas activées par défaut dans \texttt{fontspec}, ceci est particulièrement utiles par exemple pour saisir les tirets de citation \texttt{--} et \texttt{---} au lieu de saisir les vrais caractères directement). Cette option devrait toujours être utilisée.
  \itemperso{\texttt{[Numbers=OldStyle]}}Ceci permet d'avoir les version minuscules des chiffres.
  \itemperso{\texttt{[Scale=MatchLowercase]}}Ceci permet de mettre à dimension la police. Par exemple, les polices sans empattements et à largeur fixe pourront ainsi avoir des dimensions similaires à la version roman. Une autre option possible est \texttt{MatchUppercase} ; il est aussi possible d'utiliser une valeur numérique pour la mise à l'échelle.
\end{itemize}

Si vous définisser de manière séparée une police roman, largeur fixe et sans empattements (ou si vous utilisez différentes familles de polices), il peut être utile de spécifier des paramètres pour toutes les polices. Ceci peut être exécuté avec la commande \texttt{\textbackslash defaultfontfeatures} :

\begin{lstlisting}[language={[LaTeX]TeX}]
  \defaultfontfeatures{Ligatures=TeX} % application à toutes les styles personnalisés
\end{lstlisting}

\subsubsection{Trouver les noms des polices}
Si vous désirez utiliser les polices installées sur votre système, vous pouvez utiliser le nom de la police tel qu'il apparaît dans toutes les applications de votre système. Pour Windows, ces polices sont situées dans \texttt{\textbackslash Windows\textbackslash Fonts} ; pour MAC, elles sont dans \texttt{/Library/Fonts} et pour Linux elles sont dans \texttt{/usr/local/share/fonts}. Par exemple:

\begin{lstlisting}[language={[LaTeX]TeX}]
  \setmainfont{Linux Libertine O}
\end{lstlisting}

À noter que si une police présente un espace dans son nom, il doit également être utilisé quand on décide l'utiliser avec \texttt{fontspec}.

\subsubsection{Restreindre le périmètre d'utilisation d'une police}
Il est toujours possible de restreindre la partie du document sur laquelle on réalise un changement de police. Pour cela, on utiliser des groupes qui sont définis par des parenthèses :

\begin{lstlisting}[language={[LaTeX]TeX}]
  {\myfont ...}
\end{lstlisting}

On peut également (pour plus de visibilité), définir des groupes sans les accolades :

\begin{lstlisting}[language={[LaTeX]TeX}]
  \begingroup
     \myfont ...
  \endgroup
\end{lstlisting}

Si vous pensez utiliser souvent cette fonctionnalité, on peut même créer un environnement qui le permet :

\begin{lstlisting}[language={[LaTeX]TeX}]
  \newenvironment{myfont}{\myfont}{\par}

  \begin{myfont}
    Some text in the new font.
  \end{myfont}
\end{lstlisting}

Il est enfin possible de définir une commande pour standardiser le changement de polices, de la même manière que \texttt{\textbackslash textrm} ou \texttt{textbackslash textsf},  mais avec une police personnalisée :

\begin{lstlisting}[language={[LaTeX]TeX}]
  \DeclareTextFontCommand{\textmyfont}{\myfont}

  Texte avec police par défaut. \textmyfont{Nouvelle police.} De nouveau celle par défaut.
\end{lstlisting}

\subsubsection{Quelques exemples}
\begin{minipage}{.6\textwidth}
\begin{lstlisting}[language={[LaTeX]TeX}]
  \usepackage{fontspec} % à mettre avant tout autre package sur les polices

  \newfontfamily\aniron{Aniron}
  \newfontfamily\arial{Arial}
  \newfontfamily\arcena{AR CENA}

  \begin{document}
  {\arial Bonjour à la police Arial.}
  {\aniron Bonjour à la police Aniron.}
  {\arcena Bonjour à la police AR Cena Moyen.}
  \end{document}
\end{lstlisting}
\end{minipage}
\hfill%
\begin{minipage}{.35\textwidth}
\noindent{\arial Test police Arial.}
\newline
{\aniron Test police Aniron.}
\newline
{\arcena Test police AR Cena.}
\end{minipage}
\hfill\rule{0pt}{0pt}

%%% Local Variables:
%%% mode: latex
%%% TeX-master: "../../Fiches_techniques_LaTeX"
%%% End:
